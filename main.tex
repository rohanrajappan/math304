\documentclass{article}
\usepackage{graphicx} % Required for inserting images
\usepackage{enumitem}
\usepackage{amsmath}
\usepackage{array}
\usepackage{amssymb}
\usepackage{tikz}
\usepackage{mathtools}
\usepackage{wasysym}
\DeclarePairedDelimiter{\ceil}{\lceil}{\rceil}
\DeclarePairedDelimiter{\floor}{\lfloor}{\rfloor}
\makeatletter
\usepackage{titling}
\newcommand{\subtitle}[1]{%
  \posttitle{%
    \par\end{center}
    \begin{center}\large#1\end{center}
    \vskip0.5em}%
}
\makeatletter
\newcommand\setItemnumber[1]{\setcounter{enum\romannumeral\@enumdepth}{\numexpr#1-1\relax}}
\makeatother
\title{MATH 302: Homework 5}
\subtitle{Case Western Reserve University}
\author{Rohan Rajappan \\rohan@case.edu}
\date{March 18, 2024}

\begin{document}

\maketitle

\section{Section 4.5}
 \paragraph{Problems: 14, 21, 24, 38}
 \begin{enumerate}
     \setItemnumber{14}
     \item $``$If today is Tuesday, what day of the week will it be 1,000 days from today?"

     Since there are 7 days in a week, we calculate the remainder of $1000/7$ or $1000\mod{7} = 6$ Thus 6 days after Tuesday is \textit{Monday}
     \setItemnumber{21}
     \item $``$Suppose $b$ is any integer. If $b \mod{12} = 55$, what is $8b \mod{12}$? In other words, if division of $b$ by 12 gives a remainder of 5, what is the remainder when $8b$ is divided by 12? Your solution should show that you obtain the same answer no matter what integer you start with."

     We can rewrite $b$ as $b = 12*k + 55$ where $k$ is any non-negative integer. Thus if we multiply $b$ by 8, we get $8b = 8(12*k + 55) = 96k + 440 = 96k+432+8 = 12(8k+36)+8$. We can denote $8k+36$ as an arbitrary integer, $q$. Thus, $8b = 12*q + 8$ and $8b\mod{12} = 8$.

     \setItemnumber{24}
     \item $``$Prove that for all integers $m$ and $n$, if $m \mod5 = 2$ and $n \mod{5}=1$ then $mn \mod{5}=2$."

     We begin by rewriting $m$ and $n$ as $m=5j+2, n=5k+1$ where $j$ and $k$ are arbitrary, non-negative integers. 
     Thus we can rewrite $mn$ as $mn=(5j+2)(5k+1) = 5jk+5j+5k+2 = 5(jk+j+k) + 2$. Since $j$ and $k$ are arbitrary integers, we can simplify this to $mn=5q+2$. Where $q$ is a non-negative integer. By definition, this can be rewritten and proves that $mn\mod{5}=2$.

     \setItemnumber{38}
     \item \textbf{Prove:} $``$For every integer $m$, $m^2=5k$, or $m^2=5k+1$, or $m^2=5k+4$ for some integer $k$."

     We can rewrite $m$ as $m=5q+r$ where $q$ is any integer and $0\leq r<5$. Thus, we can calculate $m^2 = 25q^2+10qr+r^2 = 5(5q^2+2qr)+r^2$ We can rewrite this as
     $m^2=5s+r^2$ where $s = 5q^2+2qr = $ any positive integer. The last step is to figure out possible values for r. Since $r\in\mathbb{Z}, 0\leq r<5$,
     the only options are $r=0,1,2,3,4$ 
     
     If $r=0,1,2$, then $m^2 = 5q, 5q+1, 5q+4$, respectively. (Case 1)

     If $r=3$, then $m^2=5q+9 = 5q+5+4=5(q+1)+1 \equiv 5k+1$. (Case 2)

     Lastly, if $r=4$, then $m^2=5q+16 = 5q+15+1 = 5(q+3)+1 \equiv 5k+1$. (Case 3).

     Therefore, for every integer $m$, $m^2=5k$, or $m^2=5k+1$, or $m^2=5k+4$ for some integer $k$.
     
 \end{enumerate}
  
\section{Section 4.6}
\paragraph{Problems: 7, 19, 20}

\begin{enumerate}
    \setItemnumber{7}
    \item Given that $k$ is an integer, $\lceil{k+\frac{1}{2}}\rceil = k+1$ as $k<k+\frac{1}{2}<k+1$
    \setItemnumber{19}
    \item \textbf{Prove or Disprove: } $``$For every real number $x, \lceil x-1\rceil=\lceil x\rceil-1.$"
    We can divide this into two cases: $r\in\mathbb{Z}$ and $r\not\in\mathbb{Z}$
    
    If $r\in\mathbb{Z}$, then $\ceil{r-1} = r-1 \equiv \ceil{r}-1 = r-1 $
    
    If $r\not\in\mathbb{Z}$, then $r$ must have some non-integer component. As such, $\ceil{r} = r+1$ Thus, $\ceil{r-1} = r$. With these two equivalencies, we can state that $\ceil{r-1} = r \equiv \ceil{r} - 1  = (r+1) -1 = r$

    \setItemnumber{20}
    \item \textbf{Prove or Disprove: } $\forall x,y \in\mathbb{R}, \ceil{xy} = \ceil{x}\floor{y}$
    
    This can be disproved in the case that $x=1, y=1.1$ as $1*1.1 = 1.1$ and $\ceil{1.1} = 2$, but $\ceil{1}=\floor{1.1}=1$ and $1\neq2$
\end{enumerate}

\section{Section 4.7}
\paragraph{Problems: 4, 6, 17}
\begin{enumerate}
    \setItemnumber{4}
    \item $``$Use proof by contradiction to show that for every integer $m$, $7m+4$ is not divisible by 7"
    Suppose not and suppose that $7m+4$ is divisible by 7. Thus, $7m+4$ would have to be written in a way such that there is no remainder. This is not possible. As such, $7m+4$ is not divisible by 7. \lightning

    \setItemnumber{6}
    \item $``$There is no greatest negative real number." $\rightarrow$ There is a greatest negative real number.
    
    \textbf{Proof by Contradiction: } Suppose there is a greatest negative real number, $n$, such that $\forall x<0, x<n<0$. We can divide $n$ by two and know that $n<\frac{n}{2}<0$ As such, $n$ is not the greatest negative number. \lightning

    \setItemnumber{17}
    \item $``$For all prime numbers a, b, and c, $a^2+b^2\neq c^2$"

    \textbf{Proof by Contradiction:} Let's assume that $a^2+b^2=c^2$ for all prime numbers a, b, and c.
    Then, $c^2-b^2 = a^2 = (c-b)(c+b) = a^2$. Let $b=3, c=2$, thus $c-b < 0$ but $a^2$ must be positive. As such, this is not possible. \lightning
\end{enumerate}
\section{Section 4.8}
\paragraph{Problems: 14, 23}
\begin{enumerate}
    \setItemnumber{14}
    \item \textbf{Prove or Disprove: } The sum of any two positive irrational numbers is irrational.

    This can be easily disproved by counter-example. $\sqrt2$ and $1-\sqrt2$ are both irrational. The sum of these two numbers is $\sqrt2 + (1-\sqrt2) = 1$. Thus the
    sum of two positive, irrational numbers can be rational and this statement is disproved.
    \setItemnumber{23}
    \item Prove that for any integer, $a$, $9\not\mid(a^2-3)$ 

    By the quotient remainder rule, we can state that $9\mid(a^2-3)$ if $9k = a^2-3$ where k is any integer. We cannot rewrite this in such a way where 9 is a common divisor of 9. As such, 9 cannot divide $a^2-3$.
\end{enumerate}
\section{Section 4.10}
\paragraph{Problems: 15, 16}
\begin{enumerate}
    \setItemnumber{15}
    \item 832 and 10,933
    \begin{enumerate}
        \item $10933\mod832 = 117$
        \item $832\mod117 = 13$
        \item $117\mod13 = 0$
    \end{enumerate}
    Therefore, 13 is the GCD. 
    \item 4,131 and 2,431
    \begin{enumerate}
        \item $4131\mod2431 = 1700$
        \item $2431\mod1700 = 731$
        \item $1700\mod731 = 238$
        \item $731\mod238=17$
        \item $238\mod17=0$
    \end{enumerate}
    Therefore, the GCD is 17.
\end{enumerate}
\section{Section 5.1}
\paragraph{Problems: 6, 60}
\begin{enumerate}
    \setItemnumber{6}
    \item $f_n=\floor{\frac{n}{4}}*4, n\geq1$. The first four terms are $0,0,0,4$
    \setItemnumber{60}
    \item \begin{equation}
        2*\sum_{k=1}^n (3k^2+4) + 5*\sum_{k=1}^n(2k^2-1)
        \end{equation}
        \begin{equation}
        = \sum_{k=1}^n (2(3k^2+4) + 5*(2k^2-1))
    \end{equation}
\end{enumerate}
\end{document}
