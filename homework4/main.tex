\documentclass{article}
\usepackage{graphicx} % Required for inserting images
\usepackage{enumitem}
\usepackage{amsmath}
\usepackage{array}
\usepackage{amssymb}
\usepackage{tikz}
\usepackage{titling}
\newcommand{\subtitle}[1]{%
  \posttitle{%
    \par\end{center}
    \begin{center}\large#1\end{center}
    \vskip0.5em}%
}
\makeatletter
\newcommand\setItemnumber[1]{\setcounter{enum\romannumeral\@enumdepth}{\numexpr#1-1\relax}}
\makeatother
\title{MATH 302: Homework 4}
\subtitle{Case Western Reserve University}
\author{Rohan Rajappan \\rohan@case.edu}

\date{February 19, 2024}

\begin{document}

\maketitle

\section{4.1}
 \paragraph{Problems: 9, 11, 25}
 \begin{enumerate}
     \setItemnumber{9}
     \item \textbf{Prove:} There is a real number $x$ such that $x>1$ and $2^x > x^{10}$

     Since this is an existential statement, we only need to prove one case of this being true. One number that works in this statement is$x=100$ as $100>1$ and $2^{100}>100^{10}$. Thus we have proven that $\exists n \in \mathbb{R}$ such that $n>1 \land 2^n>n^{10}$
     \setItemnumber{11}
     \item \textbf{Prove:} There is an integer $n$ such that $2n^2-5n+2$ is prime. This is an existential statement, so we only need to prove one case of $n$ existing. If we let $n=0$, our expression becomes $2(0)^2-5(0)+2$. This simplifies to $2$ and $2\in\mathbb{P}$. Thus we have proven that $\exists n$ such that $2n^2-5n+2 \in \mathbb{P}$ by using the case that $n=0$

     \setItemnumber{25}
     \item For all integers $m$ and $n$, if $mn=1$ then $m=n=1$ or $m=n=-1$.
     \begin{enumerate}
         \item \textbf{Rewrite into if-then form:} If the product of two integers is 1, then both numbers must be 1 or both numbers must be -1.
         \item \textbf{First statement:} Let $m, n \in\mathbb{Z}$ and $mn=1$ \\
         \textbf{Last statement:} Thus, $m=n=1$ or $m=n=-1$
     \end{enumerate}
 \end{enumerate}
 
  
\section{4.2}
\paragraph{Problems: 13, 27}
\begin{enumerate}
    \setItemnumber{13}
    \item \textbf{Disprove: }There exists an integer n such that $6n^2+27$ is prime. \\
    \begin{enumerate}
        \item To disprove this, we must prove that $\forall n\in\mathbb{Z}, 6n^2+27\not\in\mathbb{P}$.
        \item We begin by factoring the original expression into $3(2n^2+9)$.
        \item We know that the inner part of this expression, $2n^2+9$, will always be an integer as the square of an integer is an integer and the sum of two integers is also an integer.
        \item Thus, we denote the inner expression and let $k = 2n^2+9$. We can thus rewrite the original expression as $3k$ where $k\in\mathbb{Z}$. 
        \item We also know that $k\geq11$. The $2n^2$ part of k must be at least 2 because any negative number becomes positive and $2*1^2$ = 2. Adding 9 to that minimum, we find that $k\geq$ 11.
        \item If $k$ is any integer where $k\geq 11$, then $3k$ is the final expression. Since $3k$ is the final expression, we know that it cannot be a prime number as our final expression will always be divisible by 3.
        \item Thus, we have proven that $\forall n \in\mathbb{Z}, 6n^2+27\not\in\mathbb{P}$.
    \end{enumerate}
    \setItemnumber{27}
    \item \textbf{Prove or Disprove: }The difference of any two odd integers is even.
    \begin{enumerate}
        \item Let m and n be any two odd integers. Thus, we let $m = 2r+1$ and $k=2s+1$ where $r$ and $s$. We must prove that $m-n$ is even. In other words, we must prove that $\frac{m-n}{2} \in\mathbb{Z}$.
        \item First, we can rewrite $m-n$ as $2r+1-(2s+1) = 2r+1-2s-1 = 2r-2s$
        \item We can then factor out a 2 to state that $m-n = 2(r-s)$. We can thus determine that $m-n$ will always be a multiple of 2 and is thus divisible by 2.
        \item As such, we now know that $\forall m,n\in\mathbb{Z}$, where m and n are odd, $\frac{m-n}{2} \in\mathbb{Z}$. In other words, the difference of any two integers is even.
    \end{enumerate}
\end{enumerate}

\section{4.3}
\paragraph{Problems: 2, 7}
\begin{enumerate}
    \setItemnumber{2}
    \item $4.6037 = \frac{46307}{10000}$
    \setItemnumber{7}
    \item $52.4672167216721\dots \rightarrow$ Let $x = 52.4672167216721\dots \rightarrow 100,000x = 5246721.672167216721\dots$ Thus, $100000x - 10x = 99990x = 5246197$. Thus, $x = \frac{5246197}{99990}$
\end{enumerate}

\section{4.4}
\paragraph{Problems: 5, 16, 21, 35}
\begin{enumerate}
    \setItemnumber{5}
    \item Is $6m(2m+10)$ divisible by 4? \\ $6m(2m+10) = 12m(m+5) = 4(3m(m+5))$ Thus, this expression is divisible by 4 as we can factor out a 4.
    \setItemnumber{16}
    \item \textbf{Prove: }For all integers $a$, $b$, and $c$, if $a\mid b$ and $a\mid c$ then $a \mid (b-c)$.\\
    \begin{enumerate}
        \item Since $a\mid c$, we can rewrite $c$ as $c = aq$ where $q$ is any integer. Likewise, we rewrite $c$ as $c=ar$ where $r$ is any integer.
        \item Thus, we rewrite  $b-c$ as $aq - ar = a(q-r)$.
        \item From this, we know that $a$ is a factor of $b-c$. As such, we know that $a(q-r)$ must be divisible by $a$, from the definition of divisibility. This is true because $q-r$ can be rewritten as any integer, $k$. Thus, we are essentially checking if $a \mid ak$ which is true from the definition of divisibility.
        \item Thus, since $ak = b-c$, we know that $a \mid (b-c)$.
    \end{enumerate}
    \setItemnumber{21}
    \item \textbf{Prove of Disprove: }The product of any two even integers is a multiple of 4.
    \begin{enumerate}
        \item Let $m$ and $n$ be any two even integers. By definition of an even integer, $m = 2r$ and $n = 2q$ for any $r, q \in\mathbb{Z}$.
        \item Thus, $mn = 2r*2q$, thus $mn = 4rq$.
        \item If we let $rq$ equal any positive integer, $k$, we can state that $k$ is divisible by 4. In other words, $4k$ is a multiple of 4.
        \item Since $4k = 4rq = mn$, we can state that $mn$ is thus a multiple of 4.
    \end{enumerate}
    \setItemnumber{35}
    \item Two athletes run a circular track at a steady pace so that the first completes one round in 8 minutes and the second in 10 minutes. If they both start from the same spot at 4 p.m., when will be the first time they return to the start together? \\ \\
    The time at which they meet will be the least common multiple of the two numbers. We can divide both numbers by two and state that they each complete half a lap in 5 and 8 minutes respectively. The least common multiple of 5 and 8 is 40. Thus, they will meet in 40 minutes. This will be runner 1's 4th lap (8/2) and runner 2's 2.5th lap (5/2). 
\end{enumerate}

\end{document}
